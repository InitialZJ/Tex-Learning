\documentclass{article}

\usepackage{ctex}
\usepackage{amsmath}
\usepackage{amssymb}

%\usepackage[left=1.25in,right=1.25in,%
%top=1in,bottom=1in]{geometry}
% or like this:
\usepackage[hmargin=1.25in,vmargin=1in]{geometry}

\begin{document}
	{\Large\textbf{第一章 \quad 线性规划}}
	
	$\S1$\quad 线性规划
	
	在人们的生产实践中,经常会遇到如何利用现有资源来安排生产,以取得最大经济
	效益的问题。此类问题构成了运筹学的一个重要分支—数学规划,而线性规划(Linear
	Programming 简记LP)则是数学规划的一个重要分支。自从1947 年G. B. Dantzig 提出
	求解线性规划的单纯形方法以来,线性规划在理论上趋向成熟,在实用中日益广泛与深
	入。特别是在计算机能处理成千上万个约束条件和决策变量的线性规划问题之后,线性
	规划的适用领域更为广泛了,已成为现代管理中经常采用的基本方法之一。
	
	1.1 \quad 线性规划的实例与定义
	
	例1 \quad 某机床厂生产甲、乙两种机床,每台销售后的利润分别为4000 元与3000 元。
	生产甲机床需用 $A$、$B$机器加工,加工时间分别为每台 2 小时和 1 小时;生产乙机床
	需用 $A$、$B$、$C$三种机器加工,加工时间为每台各一小时。若每天可用于加工的机器时
	数分别为$A$ 机器10 小时、$B$ 机器8 小时和$C$ 机器7 小时,问该厂应生产甲、乙机床各
	几台,才能使总利润最大?
	
	上述问题的数学模型:设该厂生产$x_1$ 台甲机床和$x_2$ 乙机床时总利润最大,则 $x_1$ , $x_2$
	应满足
	
	(目标函数) max 
	\begin{equation}
		z = 4x_1 + 3x_2 
	\end{equation}

	s.t.(约束条件)
	\begin{equation}\begin{equation*}
			
		\end{equation*}
		\begin{cases}
			&2x_1+x_2 \le 10	\\	
			&x_1+x_2\le8\\
			&x_2\le7		\\
			&x_1,x_2\ge0
		\end{cases}
	\end{equation}
\begin{equation*}
	\[ \[ \(  \) \] \]
\end{equation*}
	$ _{^{\frac{\sqrt{\begin{flushleft}
						\\
						\\
						\\
						
			\end{flushleft}}}{分母}}} $
	
	
	
\end{document}