\documentclass{mcmthesis}
\mcmsetup{CTeX = false,   % 使用 CTeX 套装时,设置为 true
        tcn = 2102871, problem = ABCDEF,
        sheet = true, titleinsheet = true, keywordsinsheet = true,
        titlepage = false, abstract = true}
\usepackage{newtxtext}%\usepackage{palatino}
\usepackage{lipsum}

\usepackage[ruled,linesnumbered]{algorithm2e}  
\usepackage{amsmath}
\usepackage{subcaption}
\usepackage{titlesec}
\usepackage{titletoc}

\title{Demo}

\titlecontents{section}[0cm]{\bf}{\contentslabel{1em}}{}{\titlerule*[0.7pc]{$\cdot$}\contentspage}

\begin{document}
\begin{abstract}
Use this template to begin typing the first page (summary page) of your electronic report. This template uses a 12-point Times New Roman font. Submit your paper as an Adobe PDF electronic file (e.g. 1111111.pdf), typed in English, with a readable font of at least 12-point type.

Do not include the name of your school, advisor, or team members on this or any page.

Papers must be within the 25 page limit.

Be sure to change the control number and problem choice above.
You may delete these instructions as you begin to type your report here.

Follow us @COMAPMath on Twitter or COMAPCHINAOFFICIAL on Weibo for the most up to date contest information.

\begin{keywords}
keyword1; keyword2
\end{keywords}
\end{abstract}
\maketitle
%% Generate the Table of Contents, if it's needed.
\tableofcontents
\newpage
%%
%% Generate the Memorandum, if it's needed.
%% \memoto{\LaTeX{}studio}
%% \memofrom{Liam Huang}
%% \memosubject{Happy \TeX{}ing!}
%% \memodate{\today}
%% \logo{\LARGE I'm pretending to be a LOGO!}
%% \begin{memo}[Memorandum]
%%   \lipsum[1-3]
%% \end{memo}
%%
\section{Introduction}
\subsection{Problem Background}
\ 
\indent \lipsum[1] \ref{fig:aa}
\begin{figure}[htbp]
	\small
	\centering
	\includegraphics[width=8cm]{example-image-a}
	\caption{The name of figure} \label{fig:aa}
\end{figure}

\lipsum[2]

\subsection{Restatement of the Problem}
\ 
\begin{itemize}
	\item
	\item
	\item
\end{itemize}

\subsection{Our Approach}
\ 
\indent \lipsum[3]

\begin{itemize}
	\item
	\item
	\item
\end{itemize}

\begin{Theorem} \label{thm:latex}
\LaTeX
\end{Theorem}
\begin{Lemma} \label{thm:tex}
\TeX .
\end{Lemma}
\begin{proof}
The proof of theorem.
\end{proof}

\section{General Assumptions and Model Overview}
\ 
\indent \lipsum[4]
\begin{itemize}
	\item \textbf{Assumption 1:} A
	
	$\hookrightarrow$ \textbf{Justification:} \lipsum[5]
	\item \textbf{Assumption 2:} B
	
	$\hookrightarrow$ \textbf{Justification:} \lipsum[6] \cite{1}
	\item \textbf{Assumption 3:} C
	
	$\hookrightarrow$ \textbf{Justification:} \lipsum[7]
\end{itemize}


\lipsum[8] \eqref{aa}
\begin{equation}
a^2 \label{aa}
\end{equation}

\[
  \begin{pmatrix}{*{20}c}
  {a_{11} } & {a_{12} } & {a_{13} }  \\
  {a_{21} } & {a_{22} } & {a_{23} }  \\
  {a_{31} } & {a_{32} } & {a_{33} }  \\
  \end{pmatrix}
  = \frac{{Opposite}}{{Hypotenuse}}\cos ^{ - 1} \theta \arcsin \theta
\]
\lipsum[9]

\[
  p_{j}=\begin{cases} 0,&\text{if $j$ is odd}\\
  r!\,(-1)^{j/2},&\text{if $j$ is even}
  \end{cases}
\]

\lipsum[10] \eqref{bb}

\begin{equation}
	\arcsin \theta  =
	\mathop{{\int\!\!\!\!\!\int\!\!\!\!\!\int}\mkern-31.2mu
		\bigodot}\limits_\varphi
	{\mathop {\lim }\limits_{x \to \infty } \frac{{n!}}{{r!\left( {n - r}
				\right)!}}}
	\label{bb}
\end{equation}


\section{Model Preparation}
\subsection{Notations}
\ 
\indent \lipsum[11]

\begin{table}[htbp]
	\centering
	\caption{Notations}
	\begin{tabular}{clc}
		\toprule
		Symbol & Description & Unit \\
		\midrule
		$x$ & longitude & $^{\circ}$ \\
		$y$ & latitude & $^{\circ}$ \\
		$t$ & The time from now & year \\
		\bottomrule
	\end{tabular}%
	\label{tab:Notations}%
\end{table}%

\subsection{The Data}
\ 
\indent \lipsum[12]
\subsubsection{Data Collection}
\ 
\indent \lipsum[13]
\subsubsection{Data Cleaning}
\ 
\indent \lipsum[14]

\subsection{Geographic Coordinate System}
\ 
\indent \lipsum[15]

\section{Model I: Seawater Temperature Prediction Model}
\ 
\indent \lipsum[16]
\subsection{Description of Temperature Field}
\ 
\indent \lipsum[17]
\subsection{Autoregressive Prediction Model}
\ 
\indent \lipsum[18]
\subsection{Results}
\subsubsection{Parameter Estimation}
\ 
\indent \lipsum[19]
\subsubsection{Calaculation Results}
\ 
\indent \lipsum[20]

\section{Model II: Fish Migration Prediction Model}
\ 
\indent \lipsum[21]
\subsection{Kinematics of Migration}
\ 
\indent \lipsum[22]
\subsection{Kinetics of Migration}
\ 
\indent \lipsum[23]
\subsection{Results}
\subsubsection{Estimation of $\nabla u$}
\ 
\indent \lipsum[24]
\subsubsection{Estimation of $f(\nabla u, v)$}
\ 
\indent \lipsum[25]
\subsubsection{Migration Simulation Algorithm}
\ 
\indent \lipsum[26]

\begin{algorithm} 
	\caption{The process of location change of fish}
	\label{alg:plf}
	\KwIn{$\rho(x,y,0),\beta,v_{max},u_{min},u_{max}$}
	\KwOut{$\rho(x,y,50)$}
	\For{$t=1 \to 50$} 
	{  
		The random distractor $\epsilon _t$ can be get in the process of identification of variance
		
		The dispersed $u(x, y, t)$ can be predicted based on the model ARIMA(1,1,0) and the	$u(x, y, t-1)$
		
		The continuous $u(x, y, t)$ can be get based on the linear interposition of value of the dispersed $u(x, y, t)$
		
		The continuous $\nabla u$ can be identified based on the equation (11)
		
		The continuous $v(x, y, t)$ can be calculated based on the equation (12)
		
		The location change of each fish can be calculated based on equation (7)
		
		The $\rho_t(x, y)$ of each fish can be refreshed based on the equation (8)
	}
\end{algorithm}


\subsubsection{Calaculation Results}
\ 
\indent \lipsum[27]

\begin{figure}[htbp]
	\centering
	%%----start of first subfigure ----
	\subcaptionbox{The first%
		\label{fig:subfig:a} %% label for first subfigure
	}{\includegraphics[width=4cm]{example-image-a}}
	\hspace{1in}
	%%----start of second subfigure ----
	\subcaptionbox{The second%
		\label{fig:subfig:b} %% label for second subfigure
	}{\includegraphics[width=4cm]{example-image-a}}
	\caption{The name of figure} \label{fig:bb}
\end{figure}


\section{Model III: Fishing Company Earnings Evaluation Model}
\ 
\indent \lipsum[28]
\subsection{Fishing Company Operating Model}
\subsubsection{Assessment of Fishing Costs}
\ 
\indent \lipsum[29]
\subsubsection{Assessment of Fishing Income}
\ 
\indent \lipsum[30]
\subsubsection{Assessment of Fishing Profit}
\ 
\indent \lipsum[31]
\subsection{Results}
\subsubsection{Parameter Estimation}
\ 
\indent \lipsum[32]
\subsubsection{Migration Simulation Algorithm}
\ 
\indent \lipsum[33]
\subsubsection{Calaculation Results}
\ 
\indent \lipsum[34]
\subsection{Discussion}
\subsubsection{The Management Strategies without Consider of Territorial Sea}
\ 
\indent \lipsum[35]
\subsubsection{The Management Strategies with Consider of Territorial Sea}
\ 
\indent \lipsum[36]

\section{Test the Model}
\subsection{Sensitivity Analysis}
\ 
\indent \lipsum[37]
\subsection{Robustness Analysis}
\ 
\indent \lipsum[38]

\section{Conclusion}
\subsection{Summary of Results}
\subsubsection{Result of Problem 1}
\ 
\indent \lipsum[39]
\subsubsection{Result of Problem 2}
\ 
\indent \lipsum[40]
\subsubsection{Result of Problem 3}
\ 
\indent \lipsum[41]
\subsubsection{Result of Problem 4}
\ 
\indent \lipsum[42]
\subsection{Strength}
\begin{itemize}
	\item \textbf{Applies widely}\\
	\lipsum[43]
	\item \textbf{Improve the quality of the airport service}\\
	\lipsum[44]
\end{itemize}
\subsection{Possible Improvements}
\begin{itemize}
	\item
	\item
\end{itemize}

\begin{thebibliography}{99}
\addcontentsline{toc}{section}{References}
\bibitem{1} D.~E. KNUTH   The \TeX{}book  the American
Mathematical Society and Addison-Wesley
Publishing Company , 1984-1986.
\bibitem{2}Lamport, Leslie,  \LaTeX{}: `` A Document Preparation System '',
Addison-Wesley Publishing Company, 1986.
\bibitem{3}\url{https://www.latexstudio.net/}
\end{thebibliography}

\begin{appendices}

\section{First appendix}

In addition, your report must include a letter to the Chief Financial Officer (CFO) of the Goodgrant Foundation, Mr. Alpha Chiang, that describes the optimal investment strategy, your modeling approach and major results, and a brief discussion of your proposed concept of a return-on-investment (ROI). This letter should be no more than two pages in length.

\begin{letter}{Dear, Mr. Alpha Chiang}

\lipsum[1-2]

\vspace{\parskip}

Sincerely yours,

Your friends

\end{letter}


\section{Second appendix}
Here are simulation programmes we used in our model as follow.\\

\textbf{\textcolor[rgb]{0.98,0.00,0.00}{Input matlab source:}}
\lstinputlisting[language=Matlab]{./code/mcmthesis-matlab1.m}

some more text \textcolor[rgb]{0.98,0.00,0.00}{\textbf{Input C++ source:}}
\lstinputlisting[language=C++]{./code/mcmthesis-sudoku.cpp}

\end{appendices}
\end{document}
