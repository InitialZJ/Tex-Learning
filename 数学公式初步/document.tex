\documentclass{article}

\usepackage{ctex}
\usepackage{amsmath}

\begin{document}
%	内容...
	\section{简介}
	\LaTeX{}将排版内容分为文本模式和数学模式。文本模式用于普通文本排版,数学模式用于数学公式排版。
	\section{行内公式}
	\subsection{美元符号}
	交换律是 $a+b=b+a$,如 $1+2=2+1=3$。
	\subsection{小括号}
	交换律是 \(a+b=b+a\),如 \(1+2=2+1=3\)。
	\subsection{math环境}
	交换律是 \begin{math}a+b=b+a\end{math},如 \begin{math}1+2=2+1=3\end{math}。
	\section{上下标}
	\subsection{上标}
	$3x^2 - x + 2 = 0$
	
	$3x^{3x^2 - x + 2 = 0} - x + 2 = 0$
	\subsection{下标}
	$a_0, a_1, a_2$
	\section{希腊字母}
	$\alpha$
	$\beta$
	$\gamma$
	$\epsilon$
	$\pi$
	$\omega$
	
	$\Gamma$
	$\Delta$
	$\Theta$
	$\Pi$
	$\Omega$
	
	$\alpha^2 + \beta^2 = \gamma^2$
	\section{数学函数}
	$\log$
	$\sin$
	$\cos$
	$\arcsin$
	$\arccos$
	$\ln$
	
	$\sin^2 x + \cos^2 x = 1$
	
	$y = \arcsin x$
	
	$y = \sin^{-1} x$
	
	$y = \log_2 x$
	
	$y = \ln x$
	
	$\sqrt{2}$
	$\sqrt{x^2 + y^2}$
	$\sqrt{2 + \sqrt{2}}$
	$\sqrt[4]{x}$
	\section{分式}
	大概是原体积的$3/4$
	
	大概是原体积的$\frac{3}{4}$
	
	$\frac{x}{x^2 + x +1}$
	
	$\frac{\sqrt{x - 1}}{\sqrt{x + 1}}$
	
	$\frac{1}{1 + \frac{1}{x}}$
	
	$\sqrt{\frac{x}{x^2 + x +1}}$
	\section{行间公式}	
	\subsection{美元符号}
	交换律是
	$$a+b=b+a$$
	如
	$$1+2=2+1=3$$
	\subsection{中括号}
	交换律是
	\[a+b=b+a\]
	如
	\[1+2=2+1=3\]
	\subsection{displaymath环境}
	交换律是
	\begin{displaymath}
		a + b = b + a
	\end{displaymath}
	如
	\begin{displaymath}
		1 + 2 = 2 + 1 = 3
	\end{displaymath}
	\subsection{自动编号公式equation环境}
	交换律见式\ref{eq:commutative}
	\begin{equation}
		a + b = b + a \label{eq:commutative}
	\end{equation}
	\subsection{不编号公式equation*环境}
	交换律见式\ref{eq:commutative2}
	\begin{equation*}
		a + b = b + a \label{eq:commutative2}
	\end{equation*}

	公式的编号与交叉引用也是自动实现的,大家在排版中,要习惯于引用自动化的方式处理诸如图、表、公式的编号与交叉引用。再如公式:\ref{eq:pol1}
	\begin{equation}
		x^5 - 7x^3 + 4x = 0 \label{eq:pol1}
	\end{equation}

\end{document}